% ******************************************************************************
% ************************* User Defined Commands for CAFLOW Paper *************
% ******************************************************************************

\usepackage{amsthm}


\newtheorem{theorem}{Theorem}[section]
\newtheorem{proposition}[theorem]{Proposition}
\newtheorem*{main}{Main Theorem}
\newtheorem{lemma}[theorem]{Lemma}
\newtheorem{corollary}[theorem]{Corollary}
\newtheorem*{problem}{Problem}
\theoremstyle{definition}
\newtheorem{definition}[theorem]{Definition}
\newtheorem{remark}[theorem]{Remark}
\newtheorem{assumption}[theorem]{Assumption}
\newtheorem*{notation}{Notation}

% These commands are used in the CAFLOW paper for representing vectors and samples
\newcommand{\x}{\boldsymbol{x}}
\newcommand{\y}{\boldsymbol{y}}
\newcommand{\z}{\boldsymbol{z}}
\newcommand{\el}{\boldsymbol{l}}
\newcommand{\e}{\boldsymbol{e}}
\newcommand{\h}{\boldsymbol{h}}
\newcommand{\de}{\boldsymbol{d}}
\newcommand{\s}{\boldsymbol{s}}

\newcommand{\lpi}{LPIPS}
\newcommand{\rms}{RMSE}

% ******************************************************************************
% ************************* User Defined Commands for Conditional Image Generation with Score-Based Models *************
% ******************************************************************************

\usepackage{mdframed}
\usepackage{amsmath}

\newenvironment{customthm}[1]
  {\renewcommand\thetheorem{#1}\theorem}
  {\endtheorem}

  \newenvironment{customcoll}[1]
  {\renewcommand\thecorollary{#1}\corollary}
  {\endcorollary}

\makeatletter
\DeclareRobustCommand
  \Compactcdots{\mathinner{\cdotp\mkern-2mu\cdotp\mkern-2mu\cdotp}}

\DeclareRobustCommand
  \myvdots{\vbox{\baselineskip4\p@ \lineskiplimit\z@
    \hbox{.}\hbox{.}\hbox{.}}}

\newcommand{\subalign}[1]{%
  \vcenter{%
    \Let@ \restore@math@cr \default@tag
    \baselineskip\fontdimen10 \scriptfont\tw@
    \advance\baselineskip\fontdimen12 \scriptfont\tw@
    \lineskip\thr@@\fontdimen8 \scriptfont\thr@@
    \lineskiplimit\lineskip
    \ialign{\hfil$\m@th\scriptstyle##$&$\m@th\scriptstyle{}##$\hfil\crcr
      #1\crcr
    }%
  }%
}
\makeatother

\newcommand{\norm}[1]{\left\lVert #1 \right\rVert}

% ******************************************************************************
% ****************************** Custom Margin *********************************

% Add `custommargin' in the document class options to use this section
% Set {innerside margin / outerside margin / topmargin / bottom margin}  and
% other page dimensions
\ifsetCustomMargin
  \RequirePackage[left=37mm,right=30mm,top=35mm,bottom=30mm]{geometry}
  \setFancyHdr % To apply fancy header after geometry package is loaded
\fi

% Add spaces between paragraphs
%\setlength{\parskip}{0.5em}
% Ragged bottom avoids extra whitespaces between paragraphs
\raggedbottom
% To remove the excess top spacing for enumeration, list and description
%\usepackage{enumitem}
%\setlist[enumerate,itemize,description]{topsep=0em}

% *****************************************************************************
% ******************* Fonts (like different typewriter fonts etc.)*************

% Add `customfont' in the document class option to use this section

\ifsetCustomFont
  % Set your custom font here and use `customfont' in options. Leave empty to
  % load computer modern font (default LaTeX font).
  %\RequirePackage{helvet}

  % For use with XeLaTeX
  %  \setmainfont[
  %    Path              = ./libertine/opentype/,
  %    Extension         = .otf,
  %    UprightFont = LinLibertine_R,
  %    BoldFont = LinLibertine_RZ, % Linux Libertine O Regular Semibold
  %    ItalicFont = LinLibertine_RI,
  %    BoldItalicFont = LinLibertine_RZI, % Linux Libertine O Regular Semibold Italic
  %  ]
  %  {libertine}
  %  % load font from system font
  %  \newfontfamily\libertinesystemfont{Linux Libertine O}
\fi

% *****************************************************************************
% **************************** Custom Packages ********************************

% ************************* Algorithms and Pseudocode **************************

%\usepackage{algpseudocode}


% ********************Captions and Hyperreferencing / URL **********************

% Captions: This makes captions of figures use a boldfaced small font.
%\RequirePackage[small,bf]{caption}

\RequirePackage[labelsep=space,tableposition=top]{caption}
\renewcommand{\figurename}{Fig.} %to support older versions of captions.sty


% *************************** Graphics and figures *****************************

%\usepackage{rotating}
%\usepackage{wrapfig}

% Uncomment the following two lines to force Latex to place the figure.
% Use [H] when including graphics. Note 'H' instead of 'h'
%\usepackage{float}
%\restylefloat{figure}

% Subcaption package is also available in the sty folder you can use that by
% uncommenting the following line
% This is for people stuck with older versions of texlive
%\usepackage{sty/caption/subcaption}
\usepackage{subcaption}

% ********************************** Tables ************************************
\usepackage{booktabs} % For professional looking tables
\usepackage{multirow}

%\usepackage{multicol}
%\usepackage{longtable}
%\usepackage{tabularx}


% *********************************** SI Units *********************************
\usepackage{siunitx} % use this package module for SI units


% ******************************* Line Spacing *********************************

% Choose linespacing as appropriate. Default is one-half line spacing as per the
% University guidelines

% \doublespacing
% \onehalfspacing
% \singlespacing


% ************************ Formatting / Footnote *******************************

% Don't break enumeration (etc.) across pages in an ugly manner (default 10000)
%\clubpenalty=500
%\widowpenalty=500

%\usepackage[perpage]{footmisc} %Range of footnote options


% *****************************************************************************
% *************************** Bibliography  and References ********************

%\usepackage{cleveref} %Referencing without need to explicitly state fig /table

% Add `custombib' in the document class option to use this section
\ifuseCustomBib
   \RequirePackage[square, sort, numbers, authoryear]{natbib} % CustomBib

% If you would like to use biblatex for your reference management, as opposed to the default `natbibpackage` pass the option `custombib` in the document class. Comment out the previous line to make sure you don't load the natbib package. Uncomment the following lines and specify the location of references.bib file

%\RequirePackage[backend=biber, style=numeric-comp, citestyle=numeric, sorting=nty, natbib=true]{biblatex}
%\bibliography{References/references} %Location of references.bib only for biblatex

\fi

% changes the default name `Bibliography` -> `References'
\renewcommand{\bibname}{References}


% ******************************************************************************
% ************************* User Defined Commands ******************************
% ******************************************************************************

% *********** To change the name of Table of Contents / LOF and LOT ************

%\renewcommand{\contentsname}{My Table of Contents}
%\renewcommand{\listfigurename}{My List of Figures}
%\renewcommand{\listtablename}{My List of Tables}


% ********************** TOC depth and numbering depth *************************

\setcounter{secnumdepth}{2}
\setcounter{tocdepth}{2}


% ******************************* Nomenclature *********************************

% To change the name of the Nomenclature section, uncomment the following line

%\renewcommand{\nomname}{Symbols}


% ********************************* Appendix ***********************************

% The default value of both \appendixtocname and \appendixpagename is `Appendices'. These names can all be changed via:

%\renewcommand{\appendixtocname}{List of appendices}
%\renewcommand{\appendixname}{Appndx}

% *********************** Configure Draft Mode **********************************

% Uncomment to disable figures in `draft'
%\setkeys{Gin}{draft=true}  % set draft to false to enable figures in `draft'

% These options are active only during the draft mode
% Default text is "Draft"
%\SetDraftText{DRAFT}

% Default Watermark location is top. Location (top/bottom)
%\SetDraftWMPosition{bottom}

% Draft Version - default is v1.0
%\SetDraftVersion{v1.1}

% Draft Text grayscale value (should be between 0-black and 1-white)
% Default value is 0.75
%\SetDraftGrayScale{0.8}


% ******************************** Todo Notes **********************************
%% Uncomment the following lines to have todonotes.

%\ifsetDraft
%	\usepackage[colorinlistoftodos]{todonotes}
%	\newcommand{\mynote}[1]{\todo[author=kks32,size=\small,inline,color=green!40]{#1}}
%\else
%	\newcommand{\mynote}[1]{}
%	\newcommand{\listoftodos}{}
%\fi

% Example todo: \mynote{Hey! I have a note}