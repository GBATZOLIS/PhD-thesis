%!TEX root = ../thesis.tex
%*******************************************************************************
%****************************** Notation ***************************************
%*******************************************************************************

\section*{Notation}
\label{sec:notation}

This section introduces the notations used throughout the thesis, combining statistical concepts with elements from differential and Riemannian geometry. For details on Riemannian geometry, see \cite{boothby2003introduction,carmo1992riemannian,lee2013smooth,sakai1996riemannian}.

\subsection*{Statistical Notation}
Random variables are denoted by capital letters (\(X, Y\)), with their associated probabilities written as \(P(X), P(Y)\), and their densities as \(p_X, p_Y\). Calligraphic letters (\(\mathcal{X}, \mathcal{Y}\)) represent sample spaces, while specific realizations of random variables are denoted by lowercase bold letters (\(\mathbf{x}, \mathbf{y}\)) for vectors and lowercase unbolded letters (\(x, y\)) for scalars. Given two random variables \(X\) and \(Y\), the conditional probability of \(Y\) given \(X\) is denoted by \(P(Y \mid X)\), and its density by \(p_{Y \mid X}\). Unless explicitly stated otherwise, all probability measures are assumed to be absolutely continuous with respect to the Lebesgue measure, thereby admitting a density function.

\paragraph{Simplified Statistical Notation.}  
In certain chapters, as noted at their outset, we adopt a streamlined statistical notation to improve clarity and simplify the presentation of formulas. For instance, probability distributions are denoted using only the argument of their density. For a random variable \(X_t\) with a realization \(x_t\), we write:  
\[
p(x_t) := p_{X_t}(x_t).
\]

\subsection*{Riemannian Geometry Notation}
Let \(\manifold\) denote a smooth manifold. The space of smooth functions on \(\manifold\) is denoted by \(C^\infty(\manifold)\). The \emph{tangent space} at \(\mPoint \in \manifold\), defined as the space of all \emph{derivations} at \(\mPoint\), is written as \(\tangent_\mPoint \manifold\). Elements of \(\tangent_\mPoint \manifold\), called \emph{tangent vectors}, are denoted by \(\tangentVector_\mPoint \in \tangent_\mPoint \manifold\). The \emph{tangent bundle} of \(\manifold\) is \(\tangent \manifold\), and smooth vector fields—smooth sections of the tangent bundle—are written as \(\vectorfield(\manifold) \subset \tangent \manifold\).

A smooth manifold \(\manifold\) becomes a \emph{Riemannian manifold} if equipped with a smoothly varying \emph{metric tensor field}, \((\cdot, \cdot): \vectorfield(\manifold) \times \vectorfield(\manifold) \to C^\infty(\manifold)\). This induces a \emph{(Riemannian) metric}, \(\distance_\manifold: \manifold \times \manifold \to \mathbb{R}\). The metric tensor also defines a unique affine connection, the \emph{Levi-Civita connection}, denoted by \(\nabla_{(\cdot)} (\cdot): \vectorfield(\manifold) \times \vectorfield(\manifold) \to \vectorfield(\manifold)\).

The Levi-Civita connection allows the definition of \emph{geodesics}. A geodesic between two points \(\mPoint, \mPointB \in \manifold\) is a curve \(\geodesic_{\mPoint, \mPointB}: [0,1] \to \manifold\) of minimal length connecting \(\mPoint\) to \(\mPointB\). For a tangent vector \(\tangentVector_\mPoint \in \tangent_\mPoint \manifold\), the geodesic with initial velocity \(\dot{\geodesic}_{\mPoint, \tangentVector_\mPoint}(0) = \tangentVector_\mPoint\) defines the \emph{exponential map}, \(\exp_\mPoint: \mathcal{D}_\mPoint \to \manifold\), as:
\[
\exp_\mPoint(\tangentVector_\mPoint) := \geodesic_{\mPoint, \tangentVector_\mPoint}(1),
\]
where \(\mathcal{D}_\mPoint \subset \tangent_\mPoint \manifold\) is the domain on which \(\geodesic_{\mPoint, \tangentVector_\mPoint}(1)\) is well-defined. The \emph{logarithmic map}, \(\log_\mPoint: \exp(\mathcal{D}'_\mPoint) \to \mathcal{D}'_\mPoint\), is the inverse of \(\exp_\mPoint\), valid on the domain where \(\exp_\mPoint\) is a diffeomorphism.

Finally, let \((\manifold, (\cdot, \cdot))\) be a \(\dimInd\)-dimensional Riemannian manifold, and \(\manifoldB\) a \(\dimInd\)-dimensional smooth manifold. If \(\diffeoB: \manifoldB \to \manifold\) is a diffeomorphism, the \emph{pullback metric} on \(\manifoldB\) is defined as:
\begin{equation}
(\tangentVector, \tangentVectorB)^\diffeoB := (D_{(\cdot)} \diffeoB[\tangentVector_{(\cdot)}], D_{(\cdot)} \diffeoB[\tangentVectorB_{(\cdot)}])_{\diffeoB(\cdot)},
\label{notation:pull-back-metric}
\end{equation}
where \(D_\mPoint \diffeoB: \tangent_\mPoint \manifoldB \to \tangent_{\diffeoB(\mPoint)} \manifold\) is the differential of \(\diffeoB\). This pullback metric equips \(\manifoldB\) with a Riemannian structure, \((\manifoldB, (\cdot, \cdot)^\diffeoB)\), preserving geometric properties from \((\manifold, (\cdot, \cdot))\). In this thesis, pullback mappings follow \ref{notation:pull-back-metric}, denoted with \(\diffeoB\) as a superscript, e.g., \(\distance^\diffeoB_{\manifoldB}(\mPoint, \mPointB)\), \(\geodesic^\diffeoB_{\mPoint, \mPointB}\), \(\exp^\diffeoB_\mPoint (\tangentVector_\mPoint)\), and \(\log^\diffeoB_\mPoint \mPointB\).
